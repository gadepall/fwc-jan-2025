\documentclass[12pt,a4paper]{article}
\usepackage{gvv}
\usepackage{amsmath,amssymb}
\begin{document}
\title{\underline{\textbf{2023-1}}}
\maketitle
\begin{enumerate}
\item Let $S = \{(0,1), (1,2), (3,4)\} \cup \{(0,1), (1,2), (3,4)\}$ and $T = \{0, 1, 2, 3\}$. Then which of the following statements is(are) true?
\begin{enumerate}
\item There are infinitely many functions from $S$ to $T$  
\item There are infinitely many strictly increasing functions from $S$ to $T$  
\item The number of continuous functions from $S$ to $T$ is at most $120$  
\item Every continuous function from $S$ to $T$ is differentiable  
\end{enumerate}

\item Let $T_1$ and $T_2$ be two distinct common tangents to the ellipse $E: \frac{x^2}{6} + \frac{y^2}{3} = 1 $and the parabola $P: y = \frac{x^2}{12}.$ Suppose that the tangent $T_1$ touches $P$ and $E$ at the points $A_1$ and $A_2$, respectively, and the tangent $T_2$ touches $P$ and $E$ at the points $A_4$ and $A_3$, respectively.Then which of the following statements is(are) true?
\begin{enumerate}
\item The area of the quadrilateral $A_1A_2A_3A_4$ is $35$ square units.  
\item The area of the quadrilateral $A_1A_2A_3A_4$ is $36$ square units.  
\item The tangents $T_1$ and $T_2$ meet the $x$-axis at the point $(-3,0)$.  
\item The tangents $T_1$ and $T_2$ meet the $x$-axis at the point $(6,0)$.  
\end{enumerate}

\item Let $f : [0,1] \to [0,1]$ be the function defined by $
f(x) = \frac{x^3}{3} - \frac{2x}{9} + \frac{5}{36} + \frac{17}{36}.$ Consider the square region $S = [0,1] \times [0,1]$.Let$ G = \{(x,y) \in S \mid y > f(x) \} $ be called the green region, and let $ R = \{(x,y) \in S \mid y < f(x) \} $ be called the red region.Let $ L_h = \{(x,h) \in S \mid x \in [0,1] \} $
be the horizontal line drawn at a height $h \in [0,1]$.Then which of the following statements is(are) true?
\begin{enumerate}
\item There exists an $h \in \brak{\frac{1}{4}, \frac{2}{3}}$ such that the area of the green region above the line $L_h$ equals the area of the green region below the line $L_h$.\item[(B)] There exists an $h \in \brak{\frac{1}{4}, \frac{2}{3}}$ such that the area of the red region above the line $L_h$ equals the area of the red region below the line $L_h$. 
\item There exists an $h \in \brak{\frac{1}{4}, \frac{2}{3}}$ such that the area of the green region above the line $L_h$ equals the area of the red region below the line $L_h$.  
	\item There exists an $h \in \brak{\frac{1}{4}, \frac{2}{3}}$ such that the area of the red region above the line $L_h$ equals the area of the green region below the line $L_h$.  
\end{enumerate}

\item Let $f : (0,1) \to \mathbb{R}$ be the function defined as $ f(x) = n \quad \text{if} \quad x \in \brak{\frac{1}{n+1}, \frac{1}{n}} \quad \text{where} \quad n \in \mathbb{N}
$Let $g : (0,1) \to \mathbb{R}$ be a function such that $ \int_x^{2x} \frac{t}{t} dt < g(x) < x $ for all $x \in (0,1)$.Then, $ \lim_{x \to 0} (f(x) - g(x)).$
\begin{enumerate}
\item[(A)] does \textbf{NOT} exist  
\item[(B)] is equal to $1$  
\item[(C)] is equal to $2$  
\item[(D)] is equal to $3$  
\end{enumerate}

\item Let $Q$ be the cube with the set of vertices $ \{(x_1, x_2, x_3) \mid x_1, x_2, x_3 \in \{0,1\} \} \subset \mathbb{R}^3.$ Let $F$ be the set of all twelve lines containing the diagonals of the six faces of the cube $Q$. Let $S$ be the set of all four lines containing the main diagonals of the cube $Q$; for instance, the line passing through the vertices $(0,0,0)$ and $(1,1,1)$ is in $S$.For lines $\lambda_1$ and $\lambda_2$,let $ d(\lambda_1, \lambda_2) $ denote the shortest distance between them.Then the maximum value of $ d(\lambda_1, \lambda_2), $ as $\lambda_1$ varies over $F$ and $\lambda_2$ varies over $S$, is $ \text{(Answer here)} $
\begin{enumerate}
\item $\frac{1}{6}$
\item $\frac{1}{8}$
\item $\frac{1}{3}$
\item $\frac{1}{12}$
\end{enumerate}

\item Let $ X = \{(x,y) \mid x \in \mathbb{Z}, y \in \mathbb{Z}, \frac{x^2}{8} + \frac{y^2}{20} < 1\}.$ Three distinct points $P, Q,$ and $R$ are randomly chosen from $X$.Then the probability that $P, Q,$ and $R$ form a triangle whose area is a positive integer is 
\begin{enumerate}
\item $\frac{71}{220}$
\item $\frac{73}{220}$
\item $\frac{79}{220}$
\item $\frac{83}{220}$
\end{enumerate}

\item Let $P$ be a point on the parabola $ y = ax^2, \quad \text{where} \quad a > 0.$ The normal to the parabola at $P$ meets the $x$-axis at a point $Q$.The area of the triangle $PFQ$, where $F$ is the focus of the parabola, is 120.If the slope $m$ of the normal and $a$ are both positive integers, then the pair $(a, m)$ is 
\begin{enumerate}
\item $(2, 3)$
\item ${1, 3}$
\item ${2, 4}$
\item $(3, 4)$
\end{enumerate}


\item Let $x \in \brak{\frac{\pi}{2},\frac{\pi}{2}}$ for $x \in \mathbb{R}$. Then the number of real solutions of the equation  
$ \frac{1}{1 + \cos(2x)} = 2 \tan (\tan x) $  
in the set $ \brak{\frac{3\pi}{2},\frac{\pi}{2}} \brak{ \frac{\pi}{2}, \frac{3\pi}{2}} $ is equal to:



\begin{enumerate}
\item Let $ n \geq 2 $ be a natural number and $ f :[0,1] \to \mathbb{R} $ be the function defined by
$ f(x) =
\begin{cases} 
\frac{(n-2)x}{n}, & 0 \leq x \leq \frac{1}{2n}, \\ 
\frac{(n-1) - 3x}{2(n-1)}, & \frac{1}{2n} \leq x \leq \frac{3}{4}, \\ 
\frac{4(1-x)}{n}, & \frac{3}{4} \leq x \leq 1.
\end{cases}
$
If $ n $ is such that the area of the region bounded by the curves $ x = 0 $, $ x = 1 $, $ y = 0 $, and $ y = f(x) $ is 4, then the maximum value of $ f(x) $ is:
\end{enumerate}

\begin{enumerate}
\item Let $ \overbrace{75\underbrace{57\cdots5}_{r \text{ times}}7} $ denote the \( (r+2) \)-digit number where the first and the last digits are 7, and the remaining \( r \) digits are 5. Consider the sum $S =77 + 757 + 7557 + \cdots + 75\underbrace{55\cdots5}_{98 \text{ times}}7.$ If $ S = \frac{75\underbrace{57\cdots5}_{99 \text{ times}}7}{n} $ where \( m \) and \( n \) are natural numbers less than 3000, then the value of \( m + n \) is:
\end{enumerate}

\begin{enumerate}
\item Let $ A = \brak\{ \frac{1967 + 1686 \sin \theta}{7 - 3 \cos \theta} \mid \theta \in \mathbb{R}  $. If \( A \) contains exactly one positive integer \( n \), then the value of \( n \) is:
\end{enumerate} 

\begin{enumerate}
\item Let $P$ be the plane $3x + 2y + 3z = 16$ and let 
$
S : \alpha \hat{i} + \beta \hat{j} + \gamma \hat{k}, \quad \text{where } \alpha + \beta + \gamma = 7
$
and the distance of $(\alpha, \beta, \gamma)$ from the plane is $\frac{2}{\sqrt{22}}$.
Let $\mathbf{u}, \mathbf{v}, \mathbf{w}$ be three distinct vectors in $S$ such that $|\mathbf{u} - \mathbf{v}| = |\mathbf{v} - \mathbf{w}| = |\mathbf{w} - \mathbf{u}|$. Let $V$ be the volume of the parallelepiped determined by vectors $\mathbf{u}, \mathbf{v}, \mathbf{w}$. Then the value of $\frac{80}{3} V$ is:
\end{enumerate}

\begin{enumerate}
\item Let \( a \) and \( b \) be two nonzero real numbers. If the coefficient of \( x^5 \) in the expansion of 
$
\brak{ 2 \frac{70}{27} ax + bx}^7
$
is equal to the coefficient of \( x^{-5} \) in the expansion of 
$
\brak{ \frac{1}{ax} - bx}^7,
$
then the value of \( 2b \) is :
\end{enumerate}

\item Let $\alpha, \beta$ and $\gamma$ be real numbers. Consider the following system of linear equations:
\begin{align*}
    x + 2y + z &= 7 \\
    x + \alpha z &= 11 \\
    2x - 3y + \beta z &= \gamma
\end{align*}

Match each entry in 
\textbf{List-I}
\begin{itemize}
\item (P) If $\beta = \frac{1}{2} (7\alpha - 3)$ and $\gamma = 28$, then the system has
\item (Q) If $\beta = \frac{1}{2} (7\alpha - 3)$ and $\gamma \neq 28$, then the system has
\item (R) If $\beta \neq \frac{1}{2} (7\alpha - 3)$ where $\alpha = 1$ and $\gamma \neq 28$, then the system has
\item (S) If $\beta \neq \frac{1}{2} (7\alpha - 3)$ where $\alpha = 1$ and $\gamma = 28$, then the system has
\end{itemize}

\textbf{List-II}
\begin{itemize}
    \item (1) a unique solution
    \item (2) no solution
    \item (3) infinitely many solutions
    \item (4) $x = 11, \quad y = -2, \quad z = 0$ as a solution
    \item (5) $x = -15, \quad y = 4, \quad z = 0$ as a solution
\end{itemize}

\textbf{Correct options:}
\begin{enumerate}
\item (P) $\to$ (3), (Q) $\to$ (2), (R) $\to$ (1), (S) $\to$ (4)
\item (P) $\to$ (3), (Q) $\to$ (2), (R) $\to$ (5), (S) $\to$ (4)
\item (P) $\to$ (2), (Q) $\to$ (1), (R) $\to$ (4), (S) $\to$ (5)
\item (P) $\to$ (2), (Q) $\to$ (1), (R) $\to$ (3), (S) $\to$ (3)
\end{enumerate}

\begin{tabular}{|c|c|c|c|c|c|c|}
\hline
$x_i$ & 3 & 8 & 11 & 10 & 5 & 4 \\ 
\hline
$f_i$ & 5 & 2 & 3 & 2 & 4 & 4 \\ 
\hline
\end{tabular}

Match each entry in 
\textbf{List-I} 
\begin{enumerate}
\item (P) The mean of the above data is 
\item (Q) The median of the above data is 
\item (R) The mean deviation about the mean of the above data is 
\item (S) The mean deviation about the median of the above data is 
\end{enumerate}

\textbf{list-II}
\begin{itemize}
\item 2.5
\item 5
\item 6
\item 2.7
\item 2.4
\end{itemize}

The correct option is:
\begin{enumerate}  
    \item $P \rightarrow 3, \quad Q \rightarrow 2, \quad R \rightarrow 4, \quad S \rightarrow 5$  
    \item $P \rightarrow 3, \quad Q \rightarrow 2, \quad R \rightarrow 1, \quad S \rightarrow 5$  
    \item $P \rightarrow 2, \quad Q \rightarrow 3, \quad R \rightarrow 4, \quad S \rightarrow 1$  
    \item $P \rightarrow 3, \quad Q \rightarrow 3, \quad R \rightarrow 5, \quad S \rightarrow 5$  
\end{enumerate}

\begin{enumerate}
\item Let $\ell_1$ and $\ell_2$ be the lines  
$
\mathbf{r_1} = \lambda (\hat{i} + \hat{j} + \hat{k})
$
and  
$
\mathbf{r_2} = (\hat{j} - \hat{k}) + \mu (\hat{i} + \hat{k})
$
respectively. Let $X$ be the set of all the planes $H$ that contain the line $\ell_1$. For a plane $H$, let $d(H)$ denote the smallest possible distance between the points of $\ell_2$ and $H$.
\end{enumerate} 

\item Let $H_0$ be a plane in $X$ for which $d(H_0)$ is the maximum value of $d(H)$ as $H$ varies over all planes in $X$.
\end{enumerate}

Match each entry in \textbf{List-I} to the correct entries in \textbf{List-II}.

\textbf{List-I}
\begin{enumerate}
    \item[(P)] The value of $d(H_0)$ is
    \item[(Q)] The distance of the point $(0,1,2)$ from $H_0$ is
    \item[(R)] The distance of origin from $H_0$ is
    \item[(S)] The distance of origin from the point of intersection of planes $y=z$, $x=1$ and $H_0$ is
\end{enumerate}

\textbf{List-II}
\begin{enumerate}
    \item[(1)] $\sqrt{3}$
    \item[(2)] $\frac{1}{\sqrt{3}}$
    \item[(3)] $0$
    \item[(4)] $\sqrt{2}$
    \item[(5)] $\frac{1}{\sqrt{2}}$
\end{enumerate}


The correct option is:
\begin{enumerate}
    \item[(A)] (P) $\rightarrow$ (2), (Q) $\rightarrow$ (4), (R) $\rightarrow$ (5), (S) $\rightarrow$ (1)
    \item[(B)] (P) $\rightarrow$ (5), (Q) $\rightarrow$ (4), (R) $\rightarrow$ (3), (S) $\rightarrow$ (1)
    \item[(C)] (P) $\rightarrow$ (2), (Q) $\rightarrow$ (1), (R) $\rightarrow$ (3), (S) $\rightarrow$ (2)
    \item[(D)] (P) $\rightarrow$ (5), (Q) $\rightarrow$ (1), (R) $\rightarrow$ (4), (S) $\rightarrow$ (2)
\end{enumerate} 

\begin{enumerate}
\item Let $z$ be a complex number satisfying  
$
|z|^3 + 2z^2 + 4\overline{z} - 8 = 0,
$
where $\overline{z}$ denotes the complex conjugate of $z$. Let the imaginary part of $z$ be nonzero.
\end{enumerate}

Match each entry in 
\textbf{List-I}
\begin{enumerate}
    \item (P) $|z|^2$ is equal to 
    \item (Q) $|z - \overline{z}|^2$ is equal to  
    \item (R) $|z|^2 + z + \overline{z}$ is equal to 
    \item (S) $|z+1|^2$ is equal to 
\end{enumerate}

\textbf{List-II}
\begin{enumerate}
\item 12 
\item 4
\item 8
\item 10
\item 7
\end{enumerate}

\textbf{The correct option is:}
\begin{enumerate}
    \item[(A)] (P) $\rightarrow$ (1), (Q) $\rightarrow$ (3), (R) $\rightarrow$ (5), (S) $\rightarrow$ (4)
    \item[(B)] (P) $\rightarrow$ (2), (Q) $\rightarrow$ (1), (R) $\rightarrow$ (3), (S) $\rightarrow$ (5)
    \item[(C)] (P) $\rightarrow$ (2), (Q) $\rightarrow$ (4), (R) $\rightarrow$ (5), (S) $\rightarrow$ (1)
    \item[(D)] (P) $\rightarrow$ (2), (Q) $\rightarrow$ (3), (R) $\rightarrow$ (5), (S) $\rightarrow$ (4)
\end{enumerate}
\end{document}

