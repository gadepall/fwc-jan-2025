

\documentclass{article}
 \usepackage{gvv}

\begin{document}
\title{MATHEMATICS 2022-2}
\maketitle
\begin{enumerate}  
	\item let $\alpha$ and $\beta$ be real numbers such that $-\frac{\pi}{4} < \beta < 0 < \alpha < \frac{\pi}{4}$.If $\sin(\alpha + \beta) = \frac{1}{3} and \cos(\alpha - \beta) = \frac{2}{3}$
		then the greatest integer less than or equal to  $({\frac{\sin \alpha}{\cos \beta} + \frac{\cos \beta}{\sin \alpha} + \frac{\cos \alpha}{\sin \beta} + \frac{\sin \beta}{\cos \alpha}})	^2$
is \underline{   }.



\item If $ y(x) $ is the solution of the differential equation  

xdy - $(y^2 - 4y)$ = 0, \quad \text{for } x > 0, \quad y(1) = 2,

and the slope of the curve $ y = y(x) $ is never zero, then the value of $ 10 y(\sqrt{2}) $ is \underline{   }. 





\item The greatest integer less than or equal to  

$\int_{1}^{2} \log_2 (x^3 + 1)$ \, dx + $\int_{1}^{\log_2 9} (2^x - 1)^{1/3}$ \, dx

is \underline{   }.



\item The product of all positive real values of $x$ satisfying the equation  
	
$x^{16(\log_5 x)^3 - 68 \log_5 x}$ =$ 5^{-16}$
	
is \underline{   }.



\item If 
		
		$\beta = \lim_{x \to 0} \frac{e^{x^3} - (1 - x^3)^{1/3} + \brak{( (1 - x^2)^{1/2} - 1 )} \sin x}{x \sin^2 x}$,
	
then the value of $6\beta$ is \underline{   }.



\item Let $\beta$ be a real number. Consider the matrix  
	\begin{align}
		A = \myvec{ 
\beta & 0 & 1 \\ 
2 & 1 & -2 \\ 
3 & 1 & -2 
		}
	\end{align}
If $A^7 - (\beta - 1) A^6 - \beta A^5$ is a singular matrix, then the value of $9\beta$ is \underline{   }.



\item Consider the hyperbola  
	
$\frac{x^2}{100} - \frac{y^2}{64}$ = 
	
with foci at $S$ and $S_1$, where $S$ lies on the positive x-axis. Let $P$ be a point on the hyperbola in the first quadrant. Let $\angle S P S_1 = \alpha$, with $\alpha < \frac{\pi}{2}$.  
The straight line passing through the point $S$ and having the same slope as that of the tangent at $P$ to the hyperbola intersects the straight line $S_1 P$ at $P_1$.  

Let $\delta$ be the distance of $P$ from the straight line $S P_1$, and let $\beta = S_1 P$.  
Then the greatest integer less than or equal to  
		\begin{align}
\frac{\beta \delta}{9 \sin \frac{\alpha}{2}}
		\end{align}
is \underline{   }.



\item Consider the functions $f, g : \mathbb{R} \to \mathbb{R}$ defined by  
	
$f(x) = \frac{x^2 + 5}{12}$
		
and  		
g(x) =
\begin{align} 
	2\brak{(1 - \frac{4|x|}{3} )}, & |x| \leq \frac{3}{4}, \\ 
0, & |x| > \frac{3}{4}.
\end{align}

If $\alpha$ is the area of the region  
\begin{align}
\{(x, y) \in \mathbb{R} \times \mathbb{R} \mid |x| \leq \frac{3}{4}, 0 \leq y \leq \min\{f(x), g(x)\} \},
\end{align}
then the value of $9\alpha$ is \underline{   }.

    

    \item Let PQRS be a quadrilateral in a plane, where $QR = 1$, $\angle PQR = \angle QRS = 70^\circ$, $\angle PQS = 15^\circ$, and $\angle PRS = 40^\circ$. If $\angle RPS = \theta^\circ$, $PQ = \alpha$, and $PS = \beta$, then the interval(s) that contain(s) the value of $4\alpha\beta\sin\theta^\circ$ is/are:

    \begin{enumerate}
        \item $(0, \sqrt{2})$
        \item $(1, 2)$
        \item $(\sqrt{2}, 3)$
        \item $(2\sqrt{2}, 3\sqrt{2})$
    \end{enumerate}



    \item Let 
	    
    $\alpha = \sum_{k=1}^{\infty} \sin^{2k} \brak{(\frac{\pi}{6} )}.$
	    

    Let $g: [0,1] \to \mathbb{R}$ be the function defined by 
    
    g(x) = $2^{\alpha x}$ +$ 2^{\alpha(1 - x)}$.
    
    Then, which of the following statements is/are TRUE?

    \begin{enumerate}
        \item The minimum value of $g(x)$ is $2^{7/6}$.
        \item The maximum value of $g(x)$ is $1 + 2^{1/3}$.
        \item The function $g(x)$ attains its maximum at more than one point.
        \item The function $g(x)$ attains its minimum at more than one point.
    \end{enumerate}



\item Let $\bar{z}$ denote the complex conjugate of a complex number $z$. If $z$ is a non-zero complex number for which both real and imaginary parts of 
$(\bar{z})^2 + \frac{1}{z^2}$ are integers, then which of the following is/are possible value(s) of $|z|$?

    \begin{enumerate}
	    \item $\brak{(\frac{43 + 3\sqrt{205}}{2})}^{1/4}$
	    \item $\brak{(\frac{7 + \sqrt{33}}{4})}^{1/4}$
	    \item $\brak{(\frac{9 + \sqrt{65}}{4})}^{1/4}$
	    \item $\brak{(\frac{7 + \sqrt{13}}{6})}^{1/4}$
    \end{enumerate}

    

    \item Let $G$ be a circle of radius $R > 0$. Let $G_1, G_2, \dots, G_n$ be $n$ circles of equal radius $r > 0$. Suppose each of the $n$ circles $G_1, G_2, \dots, G_n$ touches the circle $G$ externally. Also, for $i = 1, 2, \dots, n - 1$, the circle $G_i$ touches $G_{i+1}$ externally, and $G_n$ touches $G_1$ externally. Then, which of the following statements is/are TRUE?

    \begin{enumerate}
        \item If $n = 4$, then $(\sqrt{2} - 1)r < R$.
        \item If $n = 5$, then $r < R$.
        \item If $n = 8$, then $(\sqrt{2} - 1) r < R$.
        \item If $n = 12$, then $\sqrt{2} (\sqrt{3} + 1) r > R$.
    \end{enumerate}



    \item Let $\hat{i}, \hat{j}, \hat{k}$ be the unit vectors along the three positive coordinate axes. Let  
	    \begin{align}
		    \vec{\overrightarrow{a}} = 3\hat{i} + \hat{j} - \hat{k}, 
	    \end{align}
\begin{align}	    
        \vec{\overrightarrow{b}}  = \hat{i} + b_2 \hat{j} + b_3 \hat{k}, \quad b_2, b_3 \in \mathbb{R},
\end{align}

	    \begin{align}
    \vec{\overrightarrow{c}} = c_1 \hat{i} + c_2 \hat{j} + c_3 \hat{k}, \quad c_1, c_2, c_3 \in \mathbb{R}
	    \end{align}
	    
    be three vectors such that $b_2 b_3 > 0$, $\vec{\overrightarrow{a}} \cdot \vec{\overrightarrow{b}} = 0$, and  

$\myvec{
0 &-c_3& c_2\\
c_3&0&-c_1\\
-c_2&c_1&0 
}$
    
$\myvec{
1 \\
b_2 \\
b_3}
$
    =

$\myvec{
3 - c_1\\
1 - c_2 \\
-1 - c_3}
$
Then, which of the following statements is/are TRUE?
    \begin{enumerate}
	    \item $\vec{\overrightarrow{a}} \cdot \vec{c}$ = 0
	    \item $\vec{\overrightarrow{b}} \cdot \vec{\overrightarrow{c}}$ = 0
	    \item $|\vec{\overrightarrow{b}}| > \sqrt{10}$
	    \item $|\vec{\overrightarrow{c}}| \leq \sqrt{11}$
    \end{enumerate}
    


    \item For $x \in \mathbb{R}$, let the function $y(x)$ be the solution of the differential equation  
	    \begin{align}
    \frac{dy}{dx} + 12y = \cos \brak{(\frac{\pi}{12} x )}, \quad y(0) = 0.
	    \end{align}
    Then, which of the following statements is/are TRUE?

    \begin{enumerate}
        \item $y(x)$ is an increasing function.
        \item $y(x)$ is a decreasing function.
        \item There exists a real number $\beta$ such that the line $y = \beta$ intersects the curve $y = y(x)$ at infinitely many points.
        \item $y(x)$ is a periodic function.
    \end{enumerate}



    \item Consider 4 boxes, where each box contains 3 red balls and 2 blue balls. Assume that all 20 balls are distinct. In how many different ways can 10 balls be chosen from these 4 boxes so that from each box at least one red ball and one blue ball are chosen?

    \begin{enumerate}
        \item 21816
        \item 85536
        \item 12096
        \item 156816
    \end{enumerate}


 \item If 
$
M = \myvec{ 
\frac{5}{2} & \frac{3}{2} \\ 
-\frac{3}{2} & \frac{1}{2} 
}
$
then which of the following matrices is equal to \( M^{2022} \)?
\begin{enumerate}
\item $(A) \myvec{ 3034 & 3033 \\ -3033 & -3032 }$
\item $(B) \myvec{3034 & -3033 \\ 3033 & -3032 }$
\item $(C) \myvec{ 3033 & 3032 \\ -3032 & -3031 }$
\item $(D) \myvec{ 3032 & 3031 \\ -3031 & -3030}$
\end{enumerate}

\item Suppose that:\\
Box-I contains 8 red, 3 blue, and 5 green balls.\\
Box-II contains 24 red, 9 blue, and 15 green balls.\\
Box-III contains 1 blue, 12 green, and 3 yellow balls.\\
Box-IV contains 10 green, 16 orange, and 6 white balls.\\
A ball is chosen randomly from Box-I; call this ball \( b \). If \( b \) is red, then a ball is chosen randomly from Box-II. If \( b \) is blue, then a ball is chosen randomly from Box-III. If \( b \) is green, then a ball is chosen randomly from Box-IV.The conditional probability of the event "one of the chosen balls is white" given that the event "at least one of the chosen balls is green" has happened, is equal to:
\begin{enumerate}

\item $(A) \frac{15}{256} \quad$
\item $(B) \frac{3}{16} \quad$ 
\item $(C) \frac{5}{52} \quad$ 
\item $(D) \frac{1}{8}$

\end{enumerate}

\item For a positive integer \( n \), define: f(n) = n + $\frac{16 + 5n - 3n^2}{4n + 3n^2} + \frac{32+ n - 3n^2}{8n + 3n^2} + \frac{48 - 3n - 3n^2}{12n + 3n^2} + \dots + \frac{25n - 7n^2}{7n^2}$
Then, the value of \( \lim\limits_{n \to \infty} f(n) \) isequal to:
\begin{enumerate}
	\item $(A) 3 + \frac{4}{3} \log_e 7$
	\item $(B) 4 - \frac{3}{4} \log_e \frac{7}{3}$
	\item $(C) 4 - \frac{4}{3} \log_e \frac{7}{3}$
	\item $(D) 3 + \frac{3}{4} \log_e 7$
\end{enumerate}

\end{enumerate}

\end{document}

